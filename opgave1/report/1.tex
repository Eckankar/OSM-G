\section{Task 1}
\subsection{The insert function}
The idea behind the \mono{insert} function is to insert an element into
a Binary Search Tree(BST), such that the tree continues to be a BST. We've
chosen to also allow \mono{insert} to work with a null reference passed as
the first argument, allowing you to create the initial root of the tree through
the function.

The function makes sure to allocate memory for the node as appropriate, making
sure that if memory is not available an appropriate exit message is printed. The
utility function responsible for this is \mono{verify\_malloc}, 
found in \mono{util.c}.

\mono{insert} is recursive, traversing through the tree until it finds the
appropriate place that has a free leaf-node (a \mono{NULL} leaf-node), and
then inserting the data there.

\subsection{The print\_inorder function}
The \mono{print\_inorder} function is fairly straight-forward, doing an
in-order tree traversal of the \mono{tnode\_t} structure passed to it. It
takes advantage of the way the call-stack works. Since each call recursively
first makes a call to the left sub-tree, it'll keep doing so until it hits a
\mono{NULL} node (Which is an empty leaf of one of the nodes at the bottom),
print the parent, then the right child. This cascades up the call-stack, making
sure that the order printed is: Left child, self, Right child.

\subsection{The size function}
The \mono{size} function works on the same basic premise 
that \mono{print\_inorder} does, but instead of printing a message
displaying the current value of data, it simply returns the number 1 for
every non-null node.

\subsection{The to\_array function}
Like the two previous functions, \mono{to\_array} traverses the data structure
in the same exact way, with an auxilliary function \mono{dump\_to\_array}. The
auxilliary function is there, because \mono{to\_array} first sets up an int array
and allocates appropriate space for it (With the help of the \mono{size} function).
Then it calls \mono{dump\_to\_array}, which takes in an \mono{tnode\_t*} and
\mono{int**}. The integer array is incremented at each call as appropriate, which
moves the pointer to the next space that needs to be set.
