\section{Implementation of the process abstraction}

\subsection{Processes and process table}

We have chosen to implement our process structure as the struct seen in
\autoref{process-t}.

\lstset{caption=The process structure.,label=process-t,float}
\begin{lstlisting}
typedef struct {
    // The name of the process.
    char name[MAX_NAME_LENGTH];
    // The return value of the process.
    // Only relevant if the state is PROCESS_DYING.
    uint32_t retval;
    // The state of the process.
    // Note that the values of all other fields are
    // garbage if the state is PROCESS_SLOT_AVAILABLE.
    process_state_t state;
} process_t;
\end{lstlisting}

We found the need for three pieces of data; the name of the
process, the return value, and the state it's currently in, represented
by an enum shown in \autoref{process-state-t}.

\lstset{caption=The process state enum.,label=process-state-t,float}
\begin{lstlisting}
typedef enum {
    PROCESS_RUNNING,
    PROCESS_DYING,
    PROCESS_SLOT_AVAILABLE
} process_state_t;
\end{lstlisting}

Since dynamic memory allocation isn't optimal in our kernel, we use a
static array of these process structs as our process table.

\subsection{Functions for working with processes}

We've implemented a few functions to assist in encapsulating the
treatment of processes. These are outlined below.

\subsubsection{\mono{process\_init}}
Before anything else, we need to initialize the process table. This is done by
setting the state of every process slot to \mono{PROCESS\_SLOT\_AVAILABLE} and
resetting our spinlock.

This is called from the \mono{init} function in \mono{init/main.c}.

\subsubsection{\mono{process\_obtain\_slot}}
Because of how launching the main process works, we chose to make obtaining a
process slot a separate function. Alternatively, one could have spawned a new
process in a new thread and killed off the old thread.

Obtaining a slot is performed by scanning through the table, finding the first
available slot and marking it as running. This is naturally safeguarded by
disabling interrupts and spinlocking.

If no process slot is found, we choose to kernel panic. Alternively, we could have
used the sleep queue to wait for a slot to become free.

\subsubsection{\mono{process\_start}}
The \mono{process\_start} function is already provided by Buenos, but we found a need to
modify it. Threads must have their process id set, and the only way to modify it
is from within the thread itself. This means you can't set the PID of the thread
from within \mono{process\_spawn}. The only way, then, to return the process id of
a newly spawned process, is to let the \mono{process\_start} function accept a PID
as an argument. It is then responsible for looking up what to execute itself.

\subsubsection{\mono{process\_spawn}}
Spawning a process is done in the obvious way, by obtaining a process id
through \mono{process\_obtain\_slot}, then spawning a new thread that launches
\mono{process\_start}.

Or, rather, a shim that converts the \mono{process\_id\_t} to a \mono{uint32\_t},
as casting function pointers isn't allowed. One could have let \mono{process\_start}
take a \mono{uint32\_t}, rather than a \mono{process\_id\_t} to avoid this, but
it seems very ugly to have a function that takes an integer as a process id, when
we have a type dedicated to representing these.

\subsubsection{\mono{process\_get\_current\_process}}
This function simply queries the current thread for its process id.

\subsubsection{\mono{process\_join}}
\mono{process\_join} uses the sleep queue to ensure that we can join on living
processes, as well as dead ones. We follow the method outlined in the Buenos
Roadmap, section 5.2.1 for sleeping on a resource.

The implementation can be seen in \autoref{process-join}.

The resource we've chosen to sleep on is the address of the process slot, as it
seems more likely to be unique than the process id, which simply is the number of
the entry in the table.

\lstset{caption=\mono{process\_join}.,label=process-join,float}
\begin{lstlisting}
uint32_t process_join(process_id_t pid) {
    interrupt_status_t intr_status;
    uint32_t retval;

    // Disable interrupts and acquire resource lock
    intr_status = _interrupt_disable();
    spinlock_acquire(&process_table_slock);

    // Sleep while the process isn't in its "dying" state.
    while(process_table[pid].state != PROCESS_DYING) {
        sleepq_add(&process_table[pid]);
        spinlock_release(&process_table_slock);
        thread_switch();
        spinlock_acquire(&process_table_slock);
    }

    retval = process_table[pid].retval;
    process_table[pid].state = PROCESS_SLOT_AVAILABLE;

    // Restore interrupts and free our lock
    spinlock_release(&process_table_slock);
    _interrupt_set_state(intr_status);
    return retval;
}
\end{lstlisting}

\subsubsection{\mono{process\_finish}}
Terminanting a process is accomplished by marking the process as dying and
signalling the sleep queue, to allow for it to be joined on.

Naturally, we lock this down by disabling interrupts and using our spinlock
while working on our process table.

\FloatBarrier % Stop floats from going any further.
