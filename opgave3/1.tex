\section{Implementing \mono{lock\_t}}

The \mono{lock\_t} structure contains a spinlock and a condition,
namely whether it is locked or not. This allows threads to wait on
the 'locked' variable. The spinlock is there to assure atomic access
to reading/writing \mono{lock\_t->locked}. \autoref{lockt} details the struct
as defined in \mono{kernel/lock\_cond.h}.

On acquiring / releasing a lock, interrupts are disabled to prevent
interrupts messing things up.

\lstset{caption=The \mono{lock\_t} structure.,label=lockt,float}
\begin{lstlisting}
typedef struct {
	spinlock_t spinlock;
	int locked;
} lock_t;
\end{lstlisting}

\FloatBarrier % Stop floats from going any further.
