\section{\mono{syscall\_fork}}

We have followed the implementation outlined in \mono{flertraadning.txt} for
implementing this.

Most of the implementation is taken directly from there, with the exception
of a slightly better looking format for the syscall handler, as shown in
\autoref{syscallhandler}, and completion of the outlined \mono{process\_finish},
as seen in \autoref{procfin}

\lstset{caption=The syscall handler for \mono{SYSCALL\_FORK},
        label=syscallhandler,float}
\begin{lstlisting}
case SYSCALL_FORK:
    RET_REG(0) = syscall_fork((void (*)(int))ARG_REG(1), ARG_REG(2));
    break;
\end{lstlisting}
\lstset{caption=\mono{process\_finish},
        label=procfin,float}
\begin{lstlisting}
/**
 * Terminates the current process and sets a return value
 */
void process_finish(uint32_t retval) {
    interrupt_status_t intr_status;
    process_id_t pid;
    thread_table_t *my_thread;

    // Find out who we are.
    pid = process_get_current_process();
    my_thread = thread_get_current_thread_entry();

    // Ensure that we're the only ones touching the process table.
    intr_status = _interrupt_disable();
    spinlock_acquire(&process_table_slock);

    // Mark the stack as free so new threads can reuse it.
    process_free_stack(my_thread);

    if(--process_table[pid].threads == 0) {
        // Last thread in process; now we die.

        // Mark ourself as dying.
        process_table[pid].retval = retval;
        process_table[pid].state = PROCESS_DYING;

        vm_destroy_pagetable(my_thread->pagetable);

        // Wake whomever may be sleeping for the process
        sleepq_wake(&process_table[pid]);
    }

    // Free our locks.
    spinlock_release(&process_table_slock);
    _interrupt_set_state(intr_status);

    my_thread->pagetable = NULL;

    // Kill the thread.
    thread_finish();
}
\end{lstlisting}
