\documentclass[9pt, a4paper]{article}

% LANGUAGE
\usepackage[utf8]{inputenc}
%\usepackage[english]{babel}
\usepackage{synttree}

\usepackage{longtable}

\usepackage{units}

% MATH
\usepackage{amssymb, amsmath}


% COLORS
\usepackage[usenames,dvipsnames]{color}

% GRAPHICS
\usepackage{graphicx}

% TABLE
\usepackage[table]{xcolor}
\definecolor{tableShade}{RGB}{245,245,245}
\usepackage{multirow, booktabs}

% LINKS
\usepackage{url}
\usepackage[pdftex,colorlinks=true]{hyperref}
\hypersetup
{
	linkcolor=RoyalBlue,
	anchorcolor=RoyalBlue,
	citecolor=RoyalBlue,
	urlcolor=RoyalBlue,
  pdftitle={OSM Assignment 1},
  pdfauthor={Martin Grunbaum and Sebastian Paaske Tørholm},
  pdfkeywords={OSM, C},
  pdfstartview={FitV},
	bookmarksnumbered
}


% FONT
\usepackage[T1]{fontenc}
\usepackage{mathpazo}

% CAPTIONS
\usepackage[hang,small,bf]{caption}

% KU FRONT
\usepackage[babel, en]{ku-forside}
\titel{Group Assignment 3}
\undertitel{Styresystemer og Multiprogrammering}
\opgave{}
\forfatter{Martin Grunbaum(31-03-88) \& Sebastian Tørholm(23-06-88)}
\dato{February 21st, 2011}
\vejleder{}

% Stopping floats
\usepackage{placeins}

%\def\Authors{Martin Grunbaum, Oleksandr Shturmov, Jan Kaihøj Søe}
%\def\authors{Martin G., Oleksandr S., Jan S.}
%\def\Binfo{31-03-1988, 17-11-1990 \& 22-09-1989}
%\def\binfo{31-03-88\ \ \ , 17-11-90 \ \ \ \ \ \& 22-09-89 }

% PDF
\usepackage{pdfpages}



% HEADERS
\usepackage{fullpage, fancyhdr, lastpage}

\pagestyle{fancy}
\lhead{\mono{Martin G., Sebastian T.}\\\mono{31-03-88, 23-06-88}}
\chead{\mono{Styresystemer og Multiprogrammering}\\\mono{Group Assignment 3}}
\rhead{\mono{DIKU}\\\mono{February 21th, 2011}}
\cfoot{\thepage/\pageref{LastPage}}
\renewcommand{\headrulewidth}{0in}
\renewcommand{\headsep}{40pt}
\renewcommand{\textheight}{640pt}


\usepackage{listings}


\lstset{
	tabsize=2,
	language=C,
%	numbers=left,
	breaklines=true,
	basicstyle=\ttfamily,
	keywordstyle=\color{RoyalBlue}
}

% USEFULL COMMANDS
\newcommand{\mono}[1]{{\ttfamily#1}}
\newcommand{\its}[1]{{\itshape#1}}
\newcommand{\cfgvar}[1]{{\underline{#1}}}
\newcommand{\HRule}{\rule{\linewidth}{0.5mm}}
\newcommand{\nref}[1]{\ref{#1}/(\pageref{#1})}

\newcommand{\nafig}[3][3in]
{
	\begin{figure}[htbp!]
	\centering
	\includegraphics[width=#1]{img/#2}
	\caption[]{#3}
	\label{fig:#2}
	\end{figure}
}

\newcommand{\nafigw}[3][\textwidth]
{
	\begin{figure*}[t!]
	\centering
	\includegraphics[width=#1]{img/#2}
	\caption[]{#3}
	\label{fig:#2}
	\end{figure*}
}

\newcommand{\natabw}[5]
{
	\begin{table*}[htbp!]
	\centering
	\rowcolors{2}{tableShade}{white}
	\begin{tabular}{#3}
	\hline
	#4\\
	\hline
	#5\\
	\hline
	\end{tabular}
	\caption[]{#2}
	\label{tab:#1}
	\end{table*}
}

\newcommand{\fun}[2]{\mono{$\proc{#1}$ #2}}
\newcommand{\fn}[1]{\mono{{\color{RoyalBlue}#1}}}



\newcommand{\screenref}[1]{Figure \nref{fig:#1} presents a screenshot.}

%\renewcommand{\thesubsection}{\alph{subsection})}
%\renewcommand{\thesubsubsection}{\roman{subsubsection}.}

%\usepackage[utf8]{inputenc}

\parindent 0pt
\parskip 10pt

% \usepackage{clrscode3e}

% COMMAND REDEFINITIONS
\renewcommand{\figurename}{Figure}

\RequirePackage{graphics}

% DOCUMENT
\begin{document}

\maketitle
\tableofcontents
\section{Implementation of the process abstraction}

% Exciting shizzle here

\section{dlist\_t}
(Diskuter din implementering af tree2dlist her, på Engelsk ;)

\section{tnode\_t with void data pointers}
(Diskuter din implementering her ;)

\section{Custom functions}

In order to better facilitate testing, a number of additional helper functions
have been added.

In order to free the memory we've consumed, the following functions have been
added:
\begin{description}
    \item[Problem 1] {
        \mono{delete\_tree} for recursively freeing the memory used by a tree
        and zeroing out the pointer to it.
    }
    \item[Problem 2] {
        \mono{remove\_and\_free}, which removes and frees the first element of
        the list, moving the pointer to the next element.
    }
    \item[Problem 3] {
        \mono{delete\_tree2} for recursively freeing the memory used by a tree
        and zeroing out the pointer to it.

        Note, that this does \emph{not} free the values, only the tree itself.
        If the tree is to be used with dynamically allocated structures, an
        elegant approach would be to pass a function pointer to a function
        for freeing the elements contained within.
    }
\end{description}

In addition, a \mono{print\_inorder2} was added, which prints the elements
contained within a tree, using a custom print function, in order to verify
the correctness of the data contained within.

%\bibliographystyle{amsalpha}
%\bibliography{./osm}

\end{document}

