\documentclass[9pt, a4paper]{article}

% LANGUAGE
\usepackage[utf8]{inputenc}
%\usepackage[english]{babel}
\usepackage{synttree}

\usepackage{longtable}

\usepackage{units}

% MATH
\usepackage{amssymb, amsmath}


% COLORS
\usepackage[usenames,dvipsnames]{color}

% GRAPHICS
\usepackage{graphicx}

% TABLE
\usepackage[table]{xcolor}
\definecolor{tableShade}{RGB}{245,245,245}
\usepackage{multirow, booktabs}

% LINKS
\usepackage{url}
\usepackage[pdftex,colorlinks=true]{hyperref}
\hypersetup
{
	linkcolor=RoyalBlue,
	anchorcolor=RoyalBlue,
	citecolor=RoyalBlue,
	urlcolor=RoyalBlue,
  pdftitle={OSM Assignment 1},
  pdfauthor={Martin Grunbaum and Sebastian Paaske Tørholm},
  pdfkeywords={OSM, C},
  pdfstartview={FitV},
	bookmarksnumbered
}


% FONT
\usepackage[T1]{fontenc}
\usepackage{mathpazo}

% CAPTIONS
\usepackage[hang,small,bf]{caption}

% KU FRONT
\usepackage[babel, en]{ku-forside}
\titel{Group Assignment 3}
\undertitel{Styresystemer og Multiprogrammering}
\opgave{}
\forfatter{Martin Grunbaum(31-03-88) \& Sebastian Tørholm(23-06-88)}
\dato{February 21st, 2011}
\vejleder{}

% Stopping floats
\usepackage{placeins}

%\def\Authors{Martin Grunbaum, Oleksandr Shturmov, Jan Kaihøj Søe}
%\def\authors{Martin G., Oleksandr S., Jan S.}
%\def\Binfo{31-03-1988, 17-11-1990 \& 22-09-1989}
%\def\binfo{31-03-88\ \ \ , 17-11-90 \ \ \ \ \ \& 22-09-89 }

% PDF
\usepackage{pdfpages}



% HEADERS
\usepackage{fullpage, fancyhdr, lastpage}

\pagestyle{fancy}
\lhead{\mono{Martin G., Sebastian T.}\\\mono{31-03-88, 23-06-88}}
\chead{\mono{Styresystemer og Multiprogrammering}\\\mono{Group Assignment 3}}
\rhead{\mono{DIKU}\\\mono{February 21th, 2011}}
\cfoot{\thepage/\pageref{LastPage}}
\renewcommand{\headrulewidth}{0in}
\renewcommand{\headsep}{40pt}
\renewcommand{\textheight}{640pt}


\usepackage{listings}


\lstset{
	tabsize=2,
	language=C,
%	numbers=left,
	breaklines=true,
	basicstyle=\ttfamily,
	keywordstyle=\color{RoyalBlue}
}

% USEFULL COMMANDS
\newcommand{\mono}[1]{{\ttfamily#1}}
\newcommand{\its}[1]{{\itshape#1}}
\newcommand{\cfgvar}[1]{{\underline{#1}}}
\newcommand{\HRule}{\rule{\linewidth}{0.5mm}}
\newcommand{\nref}[1]{\ref{#1}/(\pageref{#1})}

\newcommand{\nafig}[3][3in]
{
	\begin{figure}[htbp!]
	\centering
	\includegraphics[width=#1]{img/#2}
	\caption[]{#3}
	\label{fig:#2}
	\end{figure}
}

\newcommand{\nafigw}[3][\textwidth]
{
	\begin{figure*}[t!]
	\centering
	\includegraphics[width=#1]{img/#2}
	\caption[]{#3}
	\label{fig:#2}
	\end{figure*}
}

\newcommand{\natabw}[5]
{
	\begin{table*}[htbp!]
	\centering
	\rowcolors{2}{tableShade}{white}
	\begin{tabular}{#3}
	\hline
	#4\\
	\hline
	#5\\
	\hline
	\end{tabular}
	\caption[]{#2}
	\label{tab:#1}
	\end{table*}
}

\newcommand{\fun}[2]{\mono{$\proc{#1}$ #2}}
\newcommand{\fn}[1]{\mono{{\color{RoyalBlue}#1}}}



\newcommand{\screenref}[1]{Figure \nref{fig:#1} presents a screenshot.}

%\renewcommand{\thesubsection}{\alph{subsection})}
%\renewcommand{\thesubsubsection}{\roman{subsubsection}.}

%\usepackage[utf8]{inputenc}

\parindent 0pt
\parskip 10pt

% \usepackage{clrscode3e}

% COMMAND REDEFINITIONS
\renewcommand{\figurename}{Figure}

\RequirePackage{graphics}

% DOCUMENT
\begin{document}

\maketitle
\tableofcontents
\section{Implementation of the process abstraction}

\subsection{Processes and process table}

We have chosen to implement our process structure as the struct seen in
\autoref{process-t}.

\lstset{caption=The process structure.,label=process-t,float}
\begin{lstlisting}
typedef struct {
    // The name of the process.
    char name[MAX_NAME_LENGTH];
    // The return value of the process.
    // Only relevant if the state is PROCESS_DYING.
    uint32_t retval;
    // The state of the process.
    // Note that the values of all other fields are
    // garbage if the state is PROCESS_SLOT_AVAILABLE.
    process_state_t state;
} process_t;
\end{lstlisting}

We found the need for three pieces of data to be stored; the name of the
process, the return value, and the state it's currently in, represented
by an enum shown in \autoref{process-state-t}.

\lstset{caption=The process state enum.,label=process-state-t,float}
\begin{lstlisting}
typedef enum {
    PROCESS_RUNNING,
    PROCESS_DYING,
    PROCESS_SLOT_AVAILABLE
} process_state_t;
\end{lstlisting}

Since dynamic memory allocation isn't optimal in our kernel, we use a
static array of these process structs as our process table.

\subsection{Functions for working with processes}

We found the need for implementing a few functions, in order to help
encapsulating the treatment of processes.

\subsubsection{\mono{process\_init}}
Before anything else, we need to initialize the process table. This is done by
setting the state of every process slot to \mono{PROCESS\_SLOT\_AVAILABLE} and
resetting our spinlock.

This is called from the \mono{init} function in \mono{init/main.c}.

\subsubsection{\mono{process\_obtain\_slot}}
We found the need for externally being able to obtain a process slot; namely for
launching the main process.

Alternatively, one could have chosen to spawn a new process in a new thread
and kill off the old thread.

Obtaining a slot is performed by scanning through the table, finding the first
available slot and marking it as running. This is naturally safeguarded by
disabling interrupts and spinlocking.

If no process slot is found, we choose to kernel panic. Alternively, we could have
used the sleep queue to wait for a slot to become free.

\subsubsection{\mono{process\_start}}
This we already given, but we found we had to modify it a bit to take a
process id. This had to be done, as the thread process id only can be modified
from within the thread itself. This gave us the problem that we couldn't set the
PID of the thread from within \mono{process\_spawn}, which meant that the only
way to return the process id of the newly spawned process was to let process\_start
accept a PID as an argument.

\subsubsection{\mono{process\_spawn}}
Spawning a process is done in the obvious way, by obtaining a process id
through \mono{process\_obtain\_slot}, then spawning a new thread that launches
\mono{process\_start}.

Or, rather, a shim that converts the \mono{process\_id\_t} to a \mono{uint32\_t},
as casting function pointers isn't allowed. One could have let \mono{process\_start}
take a \mono{uint32\_t}, rather than a \mono{process\_id\_t} to avoid this, but
it seems very ugly to have a function that takes an integer as a process id, when
we have a type dedicated to representing these.

\subsubsection{\mono{process\_get\_current\_process}}
This function simply queries the current thread for its process id.

\subsubsection{\mono{process\_join}}
\mono{process\_join} uses the sleep queue to ensure that we can join on living
processes, as well as dead ones. We follow the method outlined in the Buenos
Roadmap, section 5.2.1 for sleeping on a resource.

The implementation can be seen in \autoref{process-join}.

The resource we choose to sleep on is the address of the process slot, as it
seems more likely to be unique than the process id, which simply is the number of
the entry in the table.

\lstset{caption=\mono{process\_join}.,label=process-join,float}
\begin{lstlisting}
uint32_t process_join(process_id_t pid) {
    interrupt_status_t intr_status;
    uint32_t retval;

    // Disable interrupts and acquire resource lock
    intr_status = _interrupt_disable();
    spinlock_acquire(&process_table_slock);

    // Sleep while the process isn't in its "dying" state.
    while(process_table[pid].state != PROCESS_DYING) {
        sleepq_add(&process_table[pid]);
        spinlock_release(&process_table_slock);
        thread_switch();
        spinlock_acquire(&process_table_slock);
    }

    retval = process_table[pid].retval;
    process_table[pid].state = PROCESS_SLOT_AVAILABLE;

    // Restore interrupts and free our lock
    spinlock_release(&process_table_slock);
    _interrupt_set_state(intr_status);
    return retval;
}
\end{lstlisting}

\subsubsection{\mono{process\_finish}}
Terminanting a process is accomplished by marking the process as dying and
signalling the sleep queue, to allow for it to be joined on.

Naturally, we lock this down by disabling interrupts and using our spinlock
while working on our process table.

\FloatBarrier % Stop floats from going any further.

\section{Conditional variables}

The implementation of conditional variables is pretty simple.

A conditional variable is simply a \mono{void *}, as we had no data that
needed to be stored inside of it.

When waiting on a conditional variable, the thread is simply put on the
sleep queue, and the given lock is released, to be relocked whenever it is
reawakened.

For signal and broadcast, we simply perform a \mono{sleepq\_wake} and
\mono{sleepq\_wake\_all}, respectively. As for why we're getting a lock, we're
unsure.

For resetting, we simply wake anything that may be sleeping on the conditional,
just to make sure that everything will work the way we intend it.

%\bibliographystyle{amsalpha}
%\bibliography{./osm}

\end{document}

