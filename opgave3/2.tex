\section{Implementation of selected system calls}
We were tasked with implementing \mono{syscall\_exit},
\mono{syscall\_exec}, \mono{syscall\_join}, \mono{syscall\_write}
and \mono{syscall\_read}. Due to the utility functions that we
wrote in order to deal with processes, this was a very straight-forward
task.

\subsection{\mono{syscall\_read} and \mono{syscall\_write}}
Gathering inspiration from \mono{init/main.c}, these functions
were implemented by first fetching a reference to the
\mono{YAMS\_TYPECODE\_TTY} device, and then using its
generic device to read and write from.

We initially attempted to implement these functions with
\mono{kwrite} and \mono{kread}, by altering them to return
the amount read/written. This, for obvious reasons, did not work
as we had intended!

The functions return $-1$ if you attempt to read or write to something
other than \mono{FILEHANDLE\_STDIN} \\ \mono{FILEHANDLE\_STDOUT}.

\subsection{\mono{syscall\_exit}}
This function simply needs to exit out with a given return value. The
previously written \mono{process\_finish} function is exactly what is
needed, so the function simply calls it.

\subsection{\mono{syscall\_exec}}
Like \mono{syscall\_exit}, this is simply a matter of using a function
that we already wrote - Namely, \mono{process\_spawn}.

\subsection{\mono{syscall\_join}}
The \mono{process\_join} function performs this task, and
\mono{syscall\_join} simply returns the value returned by 
\mono{process\_join}
